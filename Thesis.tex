%---------------------------------------------------------------------------%
%->> 深圳大学毕业论文模板
%---------------------------------------------------------------------------%
%- 载入模板类
\documentclass{szuthesis}% 默认形式
% \documentclass[print]{szuthesis}% 打印预览版本,可以自动生成额外的空白页用于打印
% \documentclass[fontset=windows|adobe|mac|ubuntu]{szuthesis}% 选择字库
%---------------------------------------------------------------------------%
% - 载入配置信息,包含论文封面信息、必要的package
%---------------------------------------------------------------------------%
%->> Cover information 封面信息
%---------------------------------------------------------------------------%
\classid{TP391}% 分类号
\udc{004}
\confidential{公开}% 密级
%- 注:\title包含两个参数
% \title{深圳大学\LaTeX{}模板}{}% 单行题目,第二个参数为空
\title{新巴塞尔协议风险管理理念与}{我国风险管理体系的构建}% 多行题目
%- 注:英文题目用于生成Abstract的页眉,只有一个参数
\TITLE{The Idea of Global Risk Management from the New Basle Capital Accord and the ERM's Constructing of in China}
\author{唐同学}% 论文作者
\major{统计学}% 学科专业名称
%- 注:以下两个类型支持多行
\institute{计算机与软件学院}% 院系名称单行
% \institute{某某学院\\某某实验室}% 院系名称多行
\advisor{张老师\ 教授}% 指导教师单行
% \advisor{张老师\ 教授\\王老师\ 研究员}% 指导教师多行
\DEGREE{MasterXS}% 学术硕士
% \DEGREE{MasterZY}\degree{此处填写学位类别}% 专业硕士
%---------------------------------------------------------------------------%
%->> other config
%---------------------------------------------------------------------------%
%- 添加两个命令方便输出
\DeclareRobustCommand\cs[1]{\texttt{\char`\\#1}}
\providecommand\pkg[1]{{\sffamily#1}}
%-
\addbibresource{Biblio/ref.bib}% 参考文献路径
\setlength\bibitemsep{0.0ex plus 0.2ex minus 0.2ex}% set distance between bib entrie
%-
\setcounter{tocdepth}{3}% depth for the table of contents,设为2可不显示subsubsection
\setcounter{secnumdepth}{3}% depth for section numbering, default is 2
%-
%- 某些小语种会超出版面边界,提示Overfull \hbox{}...,中英日韩无需使用(或使用宏包microtype)
% \setlength\emergencystretch{1em}
%-
%- 重新设置 equation, figure, table 的序号
%\numberwithin{equation}{section}% set enumeration level
%\renewcommand{\theequation}{\thesection\arabic{equation}}% configure the label style
%\numberwithin{figure}{section}% set enumeration level
%\renewcommand{\thefigure}{\thesection\arabic{figure}}% configure the label style
%\numberwithin{table}{section}% set enumeration level
%\renewcommand{\thetable}{\thesection\arabic{table}}% configure the label style
\counterwithout{footnote}{chapter}% footnote编号全局连续
%-
%---------------------------------------------------------------------------%
%->> Package
%---------------------------------------------------------------------------%
% -> szuthesis.cls中已经导入的包
% - etoolbox, a toolbox of programming facilities
% - geometry, for layout
% - expl3, LaTeX3 programming environment
% - array
% - ulem, underline
% - xeCJKfntef, underline for CJK
% - fancyhdr, header and footer
% - biblatex
%-
\usepackage{graphicx}
\DeclareGraphicsExtensions{.pdf,.jpg,.png,.eps,.tif,.bmp}% 默认图片格式
\graphicspath{{Image/}}% 默认图片检索路径
%-
\usepackage[format=plain,hangindent=2.0em,font={small},skip=8pt,labelsep=space]{caption}
%-
\usepackage{subcaption}% 处理子图
%-
% \usepackage[list=off]{bicaption} % 双语caption
% \DeclareCaptionOption{bi-second}[]{
%     \def\tablename{Table}%
%     \def\figurename{Figure}%
% }
% \captionsetup[bi-second]{bi-second}
%-
\usepackage[section]{placeins}% 阻止图片浮动超出当前section
%-
\usepackage{enumitem}% 列表环境功能提升
\setlist{nosep}% 默认文本行间距
% \setlist[enumerate]{wide=\parindent}% 是否悬挂对齐,不建议全局修改
% \setlist[itemize]{wide=\parindent}
%-
% \usepackage{verbatim}
%-
% \usepackage{chemfig}% draw 2D chemical structures
% \usepackage[version=4]{mhchem}% typeset chemical formulae [mhchem|chemformula]
%-
% \usepackage{microtype}% improves general appearance of the text, 启用后降低编译效率
%-
% \usepackage{pdflscape}% landscape environment, \begin{landscape} ... \end{landscape}
%-
% \usepackage[usenames,dvipsnames,svgnames,table]{xcolor}% color support
%-
% \usepackage{tikz}% automatically load pgf package, plot with tex
% \usetikzlibrary{positioning, arrows, calc, trees }%
%-
\usepackage{booktabs}% 三线表
%-
\usepackage{listings}% 代码片段
\def\lstlistingname{代码}
\lstset{%
    basicstyle=\linespread{1.2}\small, % 字体
    breaklines=true,                   % 自动换行
    frame=lines,                       % 上下的边框,可选none|single|shadowbox等
    keepspaces=true,
    showstringspaces=false,            % string的空格添加标记,defaul:true
    tabsize=2,                         % tab长度
    % stringstyle=\color{DarkViolet},
    % backgroundcolor=\color{gray!10},
    % commentstyle=\color{ForestGreen},
    % keywordstyle=\color{blue},
}
%-
\usepackage[ruled,vlined,linesnumbered,algochapter]{algorithm2e} % 算法描述
\SetAlgorithmName{算法}{算法}{}
%---------------------------------------------------------------------------%
%->> 配置数学环境
%---------------------------------------------------------------------------%
\usepackage{amsmath}
\usepackage{amssymb}
%- 符号表,参考 http://milde.users.sourceforge.net/LUCR/Math/mathpackages/amssymb-symbols.pdf
\usepackage{amsthm} % 定理引理等环境
\theoremstyle{plain}% for theorems, lemmas, propositions, etc
\newtheorem{theorem}              {定理} [chapter]
\newtheorem{axiom}      [theorem] {公理}
\newtheorem{lemma}      [theorem] {引理}
\newtheorem{corollary}  [theorem] {推论}
\newtheorem{assertion}  [theorem] {断言}
\newtheorem{proposition}[theorem] {命题}
\newtheorem{conjecture} [theorem] {猜想}
\newtheorem{assumption} [theorem] {假设}
\theoremstyle{definition}% for definitions and examples
\newtheorem{definition}           {定义} [chapter]
\newtheorem{example}              {例}   [chapter]
\newtheorem{problem}              {问题} [chapter]
\newtheorem{exercise}             {练习} [chapter]
\theoremstyle{remark}% for remarks and notes
\newtheorem*{remark}              {注}
\newtheorem*{solution}            {解}
% \usepackage{mathtools}
\usepackage{unicode-math}
%- 注:unicode-math可以配置数学公式字体,注意包冲突!
%- 已知可能存在冲突的包:amscd,amsfonts,bbm,bm,eucal,eufrak,mathrsfs
\setmathfont{XITSMath-Regular}[
    Extension=.otf, BoldFont=XITSMath-Bold, Ligatures=TeX, StylisticSet = 1,
]
\setmathfont{XITSMath-Regular}[
    Extension=.otf, range={scr,bfscr}, Ligatures=TeX, StylisticSet = 2,
]
\setmathfont{XITSMath-Regular}[
    Extension=.otf, range={cal,bfcal}, Ligatures=TeX, StylisticSet = 1,
]
% \setmathfont{XITS Math Bold}[version=bold]% for bold version % 不兼容StylisticSet=2
% \newenvironment{szumathbf}{\bfseries\mathversion{bold}}{}
\def\XITSMathFontOptions{
    Extension=.otf, BoldFont=XITSMath-Bold, Ligatures=TeX, StylisticSet = 1
}
\setmathrm{XITSMath-Regular}[\XITSMathFontOptions]
\setmathsf{XITSMath-Regular}[\XITSMathFontOptions]
\setmathtt{XITSMath-Regular}[\XITSMathFontOptions]
%-
\def\boldsymbol#1{\symbfit{#1}}
\providecommand{\Vector}[1]{\symbfit{#1}}
\providecommand{\Matrix}[1]{\symbfit{#1}}
\providecommand{\Tensor}[1]{\symbfit{#1}}
\providecommand{\Dif}{\symrm{d}}
\providecommand{\Const}[1]{\symrm{#1}}
\providecommand{\deltarm}{\symrm{\delta}}
\providecommand{\Div}{\operatorname{div}}
\providecommand{\Trace}{\operatorname{tr}}
%---------------------------------------------------------------------------%
%->> 链接,生成书签,在最后
%---------------------------------------------------------------------------%
\usepackage{hyperref}% 超链接,生成书签,[注:放在最后]
\hypersetup{% set hyperlinks
    pdfencoding=auto,% allows non-Latin based languages in bookmarks
    psdextra=true,% extra support for math symbols in bookmarks
    bookmarksnumbered=true,% put section numbers in bookmarks
    pdftitle={\szutitle},% title
    pdfauthor={\szuauthor},% author
    pdfsubject={\szutitle},% subject
    pdfstartview={FitH},% fits the width of the page to the window
    % colorlinks=true,% false: boxed links; true: colored links
    % linkcolor=black,% color of internal links
    % citecolor=blue,% color of links to bibliography
    % filecolor=blue,% color of file links
    % urlcolor=blue,% color of external links
    hidelinks,% hide links color and box
}
%---------------------------------------------------------------------------%
%->> END
%---------------------------------------------------------------------------%
%---------------------------------------------------------------------------%
%- 辅助命令,后文中的所有\include均可在此单独列出,用逗号隔开,
%- 以此只编译必要的章节,加快编译速度,待全文完成后可注释本命令,即可编译全文。
%- 也可注释部分内容,正文中所有的内容均可注释后避免其参与编译,包含maketitle等命令,
%- 执行此命令或注释后可能导致章节序号发生错误,无需担心,全文编译后即可恢复
% \includeonly{Tex/Abstract,Tex/Appendix}
% %---------------------------------------------------------------------------%
\begin{document}
%-
\maketitle% 制作封面
%-
%- 声明包含两种形式,
%- 如果参数为空则可自动生成默认声明页,也可设置参数导入签字后的扫描版PDF文件
% \makedeclaration{declaration}% 制作声明,参数为扫描版文件名,默认在Image下
\makedeclaration{}% 制作声明,自动生成
%-
\frontmatter% 初始化摘要页环境,不建议注释
%-
%---------------------------------------------------------------------------%
%->> Abstract
%---------------------------------------------------------------------------%
%-
%-> 中文摘要
%-
\begin{abstract}
    本文是深圳大学学位论文模板 szuthesis 的使用说明文档。
    主要内容为介绍 \LaTeX{} 文档类 szuthesis 的用法,以及如何使用 \LaTeX{} 快速高效地撰写学位论文。

    \keywords{深圳大学,学位论文,\LaTeX{}模板}% 中文关键词
\end{abstract}
%-
%-> 英文摘要
%-
\begin{ABSTRACT}
    This paper is a help documentation for the \LaTeX{} class szuthesis,
    which is a thesis template for the Shenzhen University.
    The main content is about how to use the szuthesis, as well as how to write thesis
    efficiently by using \LaTeX{}.

    \KEYWORDS{Shenzhen University, Thesis, \LaTeX{} Template}% 英文关键词
\end{ABSTRACT}
%---------------------------------------------------------------------------%
%-
\tableofcontents% 目录
%-
\mainmatter% 初始化正文环境,不建议注释
%-
\chapter{引言}\label{chap:intro}

\section{研究背景}\label{sec:background}

深圳大学学位论文模板 szuthesis 基于中国科学院大学学位论文模板 ucasthesis\footnote{\url{https://github.com/mohuangrui/ucasthesis}} 二次开发。
相对于 ucasthesis ,本项目进行了许多改动,关键的变化如下:

\begin{enumerate}[wide=\parindent]
    \item 修改相关配置,使其符合深圳大学学位论文印刷格式样式;
    \item 封装了更多的命令,极大化简了 Thesis.tex 中的内容;
    \item 化简了许多操作,例如化简了字体相关补丁,遵循ctex字体关于粗体和斜体的相关设定,化简了参考文献格式等;
    \item 参考了上海交通大学论文模板 SJTUThesis\footnote{\url{https://github.com/sjtug/SJTUThesis}} 的设计思路。
\end{enumerate}

考虑到许多同学可能缺乏 \LaTeX{} 使用经验,szuthesis 不仅延续了 ucasthesis 的轻易使用的特点,同时封装了更多功能,极大简化了相关逻辑。
在本文中,对用 \LaTeX{} 撰写论文的一些主要难题,如制图、制表、文献索引等,进行了详细说明,并提供了相应的代码样本,大大降低了初学者的学习成本。
所以,如果你是初学者,请不要放弃,因为同样为初学者的我,十分明白让 \LaTeX{} 简单易用的重要性,而这正是 szuthesis 所追求和体现的。

当前 szuthesis 模板完美遵循《学术学位硕士学位论文印刷格式样式》与《专业硕士学位论文印刷格式样式》中规定的学位论文撰写要求和封面设定。
兼顾操作系统 Windows、Linux、MacOS;目前仅支持 Xe\LaTeX{} 引擎;文献编译引擎biber (biblatex)。
支持中文书签、中文渲染、拷贝 PDF 中的文本到其他文本编辑器等特性。

szuthesis 的目标在于简化学位论文的撰写,充分利用 \LaTeX{} 的优点,例如,使用 git 进行版本控制,专注写作且无需关注格式,更便利的公式书写环境,支持注释等。


\section{系统要求}\label{sec:system}

szuthesis 宏包可以在目前主流的 \href{https://en.wikibooks.org/wiki/LaTeX/Introduction}{\LaTeX{}} 发行版中使用,
如 \TeX{}Live 和 MiK\TeX{}。因 C\TeX{} 套装已停止维护,\textbf{请勿使用}。
请勿混淆 C\TeX{} 套装\footnote{\url{http://www.ctex.org/CTeX}}与 C\TeX{} 宏集\footnote{\url{https://ctan.org/pkg/ctex?lang=en}}。
C\TeX{} 套装基于 Windows 下的 MiKTeX 开发,在其基础上增加了对中文的完整支持,已于 2012 年起停止维护。
而 C\TeX{} 宏集是通用 \LaTeX{} 排版框架,为中文 \LaTeX{} 文档提供了汉字支持,主要包括宏包 ctex 以及中文文档类 ctexart、 ctexbook 等。

推荐的 \LaTeX{} 发行版如下:

\begin{center}
    %\footnotesize% fontsize
    %\setlength{\tabcolsep}{4pt}% column separation
    %\renewcommand{\arraystretch}{1.5}% row space 
    \begin{tabular}{lc}
        \toprule
        操作系统         & \LaTeX{}发行版                                                                                        \\
        \midrule
        Linux 或 Windows & \href{https://www.tug.org/texlive/}{\TeX{}Live Full} 或 \href{https://miktex.org/download}{MiK\TeX{}} \\
        MacOS            & \href{https://www.tug.org/mactex/}{Mac\TeX{} Full} 或 \href{https://miktex.org/download}{MiK\TeX{}}   \\
        \bottomrule
    \end{tabular}
\end{center}

请从各软件官网下载安装程序,勿使用不明程序源。若系统原带有旧版的 \LaTeX{} 发行版并想安装新版,请\textbf{先完全卸载旧版再安装新版}。
推荐安装2019年后的版本。可能因为网络问题导致安装速度较慢,推荐在安装时无需选择额外宏包,
安装完成后添加清华源\footnote{\url{https://mirrors.tuna.tsinghua.edu.cn/help/CTAN/}},再继续安装所需宏包。

如选择部分安装,请安装后检测以下宏包知否安装,若未安装导致的BUG不易排查:

\begin{center}
    \small% fontsize
    \renewcommand{\arraystretch}{0.8}% row space 
    \begin{tabular}{ll}
        \toprule
        宏包                 & 功能                    \\
        \midrule
        xits                 & 开源Times New Roman字体 \\
        biber                & biblatex引擎            \\
        biblatex-gb7714-2015 & biblatex格式            \\
        latexmk              & 自动编译latex文档       \\
        \bottomrule
    \end{tabular}
\end{center}

安装 \LaTeX{} 发行版后,即可使用任意编辑器开始书写。
在这里推荐使用 VS Code\footnote{\url{https://code.visualstudio.com/}} 作为编辑器。一方面其可单纯的作为编辑器使用,
同时又可以搭配插件进行扩展。可以搭配 Git 进行版本控制,又可以安装 LaTeX Workshop 插件直接进行编译。
LaTeX Workshop 插件提供了诸如 Linting,Formatting,Intellisense,PDF 文件预览,公式预览,
全文大纲等诸多功能\footnote{\url{https://github.com/James-Yu/LaTeX-Workshop\#features-taster}}。
使用快捷键可以极大提高编写效率,例如使用 \lstinline!Ctrl+Alt+j! 可以快速从 tex 文本跳转到 PDF 中对应的位置,
而在PDF预览中使用 \lstinline!Ctrl+鼠标左键! 就可以快速定位对应的 tex 文本。


\section{编译}

\begin{enumerate}[wide=\parindent]
    \item 安装软件:根据所用操作系统和章节~\ref{sec:system} 中的信息安装 \LaTeX{} 编译环境。

    \item 获取模板:下载 szuthesis 项目。szuthesis 不仅提供了相应的模板,同时也提供了编译样例,
          下载时推荐下载整个 szuthesis 文件夹,而不是单独的 cls 文档类。

    \item 编译模板:参考项目主页编译部分。
\end{enumerate}

编译完成后即可获得这份说明文档。而这也完成了学习使用 szuthesis 撰写论文的一半进程。
什么?这就学成一半了,这么简单???是的,就这么简单!


\section{文档目录简介}

\subsection{Thesis.tex}

Thesis.tex 为主文档,包含了论文全篇的主要架构。其中,document 中的所有内容均可注释后避免其参与编译,包含 maketitle 等命令,
注释后可能导致章节序号发生错误,无需担心,全文编译后即可恢复。注释后可加快编译速度,例如参考文献页无须随文档实时编译,
只需要全文完成后编译参考文献页即可,这也是使用 \LaTeX{} 编写文档的优点之一。

\subsection{Temp文件夹}

编译后,生成的临时文件皆存于Temp文件夹内,包括编译得到的 PDF 文档,其存在是为了保持工作空间的整洁,因为好的心情是很重要的。

\subsection{szuthesis.cls}

\verb!texmf\tex\latex\szuthesis\szuthesis.cls! 目录下的 szuthesis.cls 为文档类,定义了论文的核心格式,
包括论文排版、引用格式、页眉页脚、字体字号等。其中,根据《印刷格式样式》规定,参考文献后的字号均与正文字号不同。

\subsection{config.tex}

szuthesis.cls 需要传入一些参数用来生成封面信息,config.tex 可用来传入这些参数。
后边则定义了一些可选的宏包,这些宏包并不完全属于《印刷格式样式》规定的排版,可以自由选择是否启用。
例如数学公式的字体、代码片段、超链接等,均在 config.tex 进行了定义,这些可以根据需要对它进行调整。

\subsection{Tex文件夹}

文件夹内为论文的所有正文内容,这也是使用 szuthesis 撰写学位论文时,主要关注和修改的一个位置。
\textbf{注:所有文件都必须采用 UTF-8 编码,否则编译后将出现乱码文本},详细分类介绍如下:

\begin{itemize}
    \item Abstract.tex:摘要信息。
    \item ChapterX.tex:论文的各个章节,可根据需要添加和撰写。\textbf{添加新章节时,注意编码格式,可拷贝一个已有的章文件再重命名,以继承文档的 UTF-8 编码}。
    \item Appendix.tex:附录,注意附录字号与正文不同,仅用于添加补充信息,如有整段文本建议放置于正文中。
    \item Acknowledgements.tex:致谢。
    \item Publications.tex: 研究成果。
\end{itemize}

\subsection{Image文件夹}

用于放置论文中所需要的图形类文件,支持格式有:jpg, png, pdf 等,需要更多支持格式可在 config.tex 中配置。
不建议为各章节图片建子目录,即使图片众多,若命名规则合理,图片查询亦是十分方便。

\subsection{Biblio文件夹}

ref.bib 为参考文献信息,可在 config.tex 中配置。

\subsection{.vscode文件夹}

这一文件夹用于保存 VS Code 的配置文件,其中 settings.json 保存了部分 latexmk 所需的配置项。



\section{帮助与问题反馈}\label{sec:help}

对于 \LaTeX{} 相关问题,推荐使用 texdoc 命令查阅相关文档。
例如安装 lshort-chinese 宏包后,可使用 \textbf{texdoc lshort-chinese} 命令打开一份教程,包含了 \LaTeX{} 入门相关的知识。
使用 \textbf{texdoc ctex} 则可打开 ctex 宏包的文档,包含中文排版相关的内容,例如第5节中则详细介绍了中文字体字号。
大多数宏包都提供了非常详尽的文档,都可以使用 texdoc 查阅。

欢迎各位同学提出宝贵意见,一起不断改进模板。

\chapter{\LaTeX{} 入门}

\section{二级标题}

\subsection{三级标题}

\subsubsection{四级标题}

深圳大学1983年经国家教育部批准设立。中央、教育部和地方高度重视特区大学建设。北大援建中
文、外语类学科,清华援建电子、建筑类学科,人大援建经济、法律类学科,一大批知名学者云集
深圳大学。建校伊始,学校在高校管理体制上锐意改革,在奖学金、学分制、勤工俭学等方面进行
了积极探索,率先在国内实行毕业生不包分配和双向选择制度,推行教职员工全员聘任制度和后勤
部门社会化管理改革,在全国引起强烈反响\footnote{这是脚注}。

\section{字体}

字体部分主要分为正文字体和数学公式字体,这里主要展示正文字体,数学公式字体详见下节。

正文字体延续 ctex 关于字体的相关设置,需要注意的是,中文字体通常无粗体与斜体,例如
Windows 平台下,正文默认宋体,粗体则使用黑体代替,斜体则使用了楷体。MacOS 对多数
字体提供了完整的粗体表示,但是默认的斜体依然使用了楷体。通常论文的正文应简洁明了,
尽量避免使用多种字体。如需使用粗体或斜体,需重新定义默认字体族或者定义新的字体族。

由上段可知,字体的选择依赖于字库,而字库又依赖于系统设置,模板支持windows、adobe、mac、ubuntu
四种中文字库,通常情况下可根据系统类型自动选择。但是 Ubuntu 系统因缺少仿宋字体,而部分章节的标题
会使用到仿宋,模板使用宋体进行了代替,依然推荐 Ubuntu 用户安装完整 adobe 字体并且指定字库。

关于字库的显示效果可参考表~\ref{tab:fontset},该表展示了 Windows 系统下主要的字体以及对应的粗体与楷体的
兼容情况。可以看到默认的粗体使用了黑体代替,而斜体则使用了楷书。中文字体的设置均不影响英文字体的设置,英文
字体除摘要页标题使用了 Lucida Console 字体外,全局使用开源 Times New Roman 字体 XITS。可以看到 \cs{textit}
和 \cs{kaishu} 都可以生成中文的楷书,但是对于英文字体却产生了不同的影响,书写时需注意。

\def\SZUFONTEG{深圳大学fgFG456}
\begin{table}
  \centering
  \caption{字体选择}\label{tab:fontset}
  \begin{tabular}{lcccc}
    \toprule
    字体 & 命令          & 常规                  & 粗体                         & 斜体                         \\
    \midrule
    宋体 & -             & {\SZUFONTEG}          & \textbf{\SZUFONTEG}          & \textit{\SZUFONTEG}          \\
         & \cs{textrm}   & \textrm{\SZUFONTEG}   & \textbf{\textrm{\SZUFONTEG}} & \textit{\textrm{\SZUFONTEG}} \\
         & \cs{songti}   & {\songti\SZUFONTEG}   & -                            & -                            \\
    黑体 & \cs{textsf}   & \textsf{\SZUFONTEG}   & \textbf{\textsf{\SZUFONTEG}} & -                            \\
         & \cs{heiti}    & {\heiti\SZUFONTEG}    & -                            & -                            \\
    仿宋 & \cs{texttt}   & \texttt{\SZUFONTEG}   & -                            & -                            \\
         & \cs{fangsong} & {\fangsong\SZUFONTEG} & -                            & -                            \\
    楷书 & \cs{textit}   & \textit{\SZUFONTEG}   & -                            & -                            \\
         & \cs{kaishu}   & {\kaishu\SZUFONTEG}   & -                            & -                            \\
    \bottomrule
  \end{tabular}
\end{table}

此外,windows 和 mac 对应的中文字库还提供了微软雅黑 \cs{yahei}、隶书 \cs{lishu}、幼圆 \cs{youyuan} 三种字体,
\cs{yahei} 默认提供了粗体。更多关于默认字体与字体库的选择相关问题可查阅 ctex 文档 4.3 节中文字库。

除了中文字库外,由 XITS 字库可支持俄语 Привет ,默认的中文字体宋体通常支持日语 こんにちは 。
如需要韩语,则需要导入新的字库,如 Windows 通常带有 ARIALUNI.TTF 韩语字库,可以使用
\lstinline[language=Tex]!\newCJKfontfamily{\hanyu}{ARIALUNI.TTF}! 导入新的字库,就可以使用 \cs{hanyu} 来
书写韩语了。其他语言同理,可以使用这个命令导入对应的字库。

\section{数学}

\subsection{数学符号与公式}

数学公式均统一使用了开源 Times New Roman 字体 XITS MATH,这套字体除了基本的数字与字母外,还额外设计了许多
数学符号,包括全局粗体。使用这套字体需导入 \pkg{unicode-math},所以在启用数学相关字体时需要注意,使用 \cs{sym..} 系列
命令代替 \cs{math..} 系列命令,以获得统一的字体风格,\cs{math..} 系列命令将会使用普通字母的字体 XITS,而不是
对应的 XITS MATH。例如加粗 \cs{mathbf} 将使用 \cs{symbf} 代替。
\[
  \begin{split}
    \symcal{L} \{f\}(s) &= \int _{0^{-}}^{\infty} f(t) e^{-st} \, \symrm{d}t, \
    \symscr{L} \{f\}(s) = \int _{0^{-}}^{\infty} f(t) e^{-st} \, \symrm{d}t\\
    \symcal{F} \bigl( f(x+x_{0}) \bigr) &= \symcal{F} \bigl( f(x) \bigr) e^{2\pi i\xi x_{0}}, \
    \symscr{F} \bigl( f(x+x_{0}) \bigr) = \symscr{F} \bigl( f(x) \bigr) e^{2\pi i\xi x_{0}}
  \end{split}
\]
\begin{equation}\label{equ:1}
  \begin{cases}
    \frac{\partial \rho}{\partial t} + \nabla\cdot(\rho\Vector{V}) = 0                                                \\
    \frac{\partial (\rho\Vector{V})}{\partial t} + \nabla\cdot(\rho\Vector{V}\Vector{V}) = \nabla\cdot\Tensor{\sigma} \\
    \frac{\partial (\rho E)}{\partial t} + \nabla\cdot(\rho E\Vector{V}) = \nabla\cdot(k\nabla T) + \nabla\cdot(\Tensor{\sigma}\cdot\Vector{V})
  \end{cases}
\end{equation}
\begin{equation}\label{equ:2}
  \frac{\partial }{\partial t}\int\limits_{\Omega} u \, \symrm{d}\Omega + \int\limits_{S} \Vector{n}\cdot(u\Vector{V}) \, \symrm{d}S = \dot{\phi}
\end{equation}

表~\ref{tab:mathfontset} 中简要示范了几种常用的数学字体,可以看到指令并不是对所有的数字和希腊字母都有效。更多关于
数学字体相关的知识推荐查阅《在 \LaTeX{} 中使用 OpenType 字体》\footnote{\url{https://stone-zeng.github.io/2020-05-02-use-opentype-fonts-iii/}\label{stonezeng}}
一文,该文中``字符支持''一节详细介绍了各个命令的使用场景。

\def\SZUMATHFONTEG{fgFG456\delta\Delta}
\begin{table}[!htbp]
  \centering
  \caption{数学公式字体选择}\label{tab:mathfontset}
  \begin{tabular}{cccccc}
    \toprule
    命令        & 样例                        & 命令         & 样例                         & 命令         & 样例                         \\
    \midrule
    -           & \({\SZUMATHFONTEG}\)        & \cs{symbf}   & \(\symbf{\SZUMATHFONTEG}\)   & \cs{symrm}   & \(\symrm{\SZUMATHFONTEG}\)   \\
    \cs{symup}  & \(\symup{\SZUMATHFONTEG}\)  & \cs{symbfup} & \(\symbfup{\SZUMATHFONTEG}\) & \cs{symit}   & \(\symit{\SZUMATHFONTEG}\)   \\
    \cs{symbb}  & \(\symbb{\SZUMATHFONTEG}\)  & \cs{symsfup} & \(\symsfup{\SZUMATHFONTEG}\) & \cs{symtt}   & \(\symtt{\SZUMATHFONTEG}\)   \\
    \cs{symcal} & \(\symcal{\SZUMATHFONTEG}\) & \cs{symscr}  & \(\symscr{\SZUMATHFONTEG}\)  & \cs{symfrak} & \(\symfrak{\SZUMATHFONTEG}\) \\
    \bottomrule
  \end{tabular}
\end{table}

在使用数学字体过程中,应遵循相应的标准,根据 GB 3102.11《物理科学和技术中使用的数学
符号》\footnote{\url{https://zh.wikisource.org/wiki/GB_3102.11}}中规定:

\begin{itemize}
  \small
  \item 变量 (例如 \(x\), \(y\) 等)、变动附标 (例如 \(\sum_i x_i\) 中的 \(i\)) 及函数 (例如 \(f\), \(g\) 等)
        用斜体字母表示。点 \(A\)、线段 \(AB\) 及弧 \(CD\) 用斜体字母表示。
        在特定场合中视为常数的参数 (例如 \(a\), \(b\) 等) 也用斜体字母表示。
  \item 有定义的已知函数 (例如 \(\sin \),\(\exp \),\(\ln \),\(\Gamma \) 等) 用正体字母表示。
  其值不变的数学常数 (例如 \(\exp = 2.718 281 8\ldots \),\(\symrm{\pi} = 3.141 592 6\ldots \),\(\symrm{i}^2 = −1\) 等) 用正体
  字母表示。已定义的算子 (例如 \(\Div \),\(\deltarm x\) 中的 \(\deltarm \) 及 \(\Dif 𝑓/\Dif 𝑥\) 中的 \(\Dif \)) 也用正体字母表示。
  \item 数字表中数 (例如 351 204,1.32,7/8) 的表示用正体。
\end{itemize}

根据这一规定,在 config.tex 中额外定义了一些通用命令,如表~\ref{tab:mathcmd} 所示。使用这些命令,
我们可以方便的定义公式,如公式\ref{equ:1}、\ref{equ:2}和公式~\ref{equ:3}\textsuperscript{\ref{stonezeng}}。
\begin{equation}\label{equ:3}
  \begin{aligned}
    \oiint_{\partial\Omega} \Vector{}{E} \cdot \Dif \Vector{S} & = \frac{1}{\epsilon_0} \iiint_\Omega \rho \, \Dif V & \quad &
    \oint_{\partial\Sigma} \Vector{E} \cdot \Dif \Vector{l} = -\frac{\Dif}{\Dif t} \iint_\Sigma \Vector{B} \cdot \Dif \Vector{S}         \\
    \oiint_{\partial\Omega} \Vector{B} \cdot \Dif \Vector{S} & = 0                                                & \quad &
    \oint_{\partial\Sigma} \Vector{B} \cdot \Dif \Vector{l} = \mu_0 \iint_\Sigma \Vector{J} \cdot \Vector{S}
    + \mu_0 \epsilon_0 \frac{\Dif}{\Dif t} \iint_\Sigma \Vector{E} \cdot \Dif \Vector{S} \\
  \end{aligned}
\end{equation}

\begin{table}[!htbp]
  \centering
  \caption{自定义指令}\label{tab:mathcmd}
  \begin{tabular}{cccccccc}
    \toprule
    用法                   & 样例            & 用法                   & 样例           & 用法                   & 样例           & 用法       & 样例        \\
    \midrule
    \verb!\Vector{A}! & \(\Vector{A}\)  & \verb!\Matrix{A}! & \(\Matrix{A}\) & \verb!\Tensor{A}! & \(\Tensor{A}\) & \cs{Dif x} & \(\Dif x\)  \\
    \verb!\Const{\pi}! & \(\Const{\pi}\) & \cs{deltarm}           & \(\deltarm \)  & \cs{Div}               & \(\Div \)      & \cs{Trace} & \(\Trace \) \\
    \bottomrule
  \end{tabular}
\end{table}

\subsection{标题中带有公式 \texorpdfstring{\(\Lambda,\lambda,\theta,\bar{\Lambda},\sqrt{S_{NN}}\)}{}}

\LaTeX{} 支持在标题中带有公式,以及生成的目录中带有公式。如本节标题所示,在目录中均可显示完整的公式。
由于 \pkg{hyperref} 在生成书签时不支持公式,所以当使用该包时,可以使用 \cs{texorpdfstring} 命令。
这个命令的第一个参数可以书写公式,第二个参数则用于生成书签,可以使用 \cs{textLambda} 等书签兼容的公式,
不过最简单的方法就是空出第二个参数,因为书签在打印时不显示,所以无需额外关注。

\subsection{数学环境}
以下环境共享一个计数器,如果取消这一设定,在 config.tex 中删除 \cs{newtheorem} 的第二个参数[theorem]即可。

\begin{theorem}      这是一个定理。 \end{theorem}
\begin{axiom}        这是一个公理。 \end{axiom}
\begin{lemma}        这是一个引理。 \end{lemma}
\begin{corollary}    这是一个推论。 \end{corollary}
\begin{assertion}    这是一个断言。 \end{assertion}
\begin{proposition}  这是一个命题。 \end{proposition}
\begin{conjecture}   这是一个猜想。 \end{conjecture}
\begin{assumption}   这是一个假设。 \end{assumption}

以下环境均为独立的计数器。

\begin{definition}   这是一个定义。 \end{definition}
\begin{example}      这是一个例子。 \end{example}
\begin{problem}      这是一个问题。 \end{problem}
\begin{exercise}     这是一个练习。 \end{exercise}
\begin{remark}       这是一个注。   \end{remark}
\begin{solution}     这是一个解。   \end{solution}
\begin{proof}        这是一个证明。 \end{proof}

\section{浮动体}

\subsection{图形}

插入单个图形时效果如图~\ref{fig:single}。同样的方法可以插入多个图形,通过可选参数可以指定图形的高度
或者宽度,效果如图~\ref{fig:multi}。如果各个子图相互独立,可以使用 \pkg{minipage},如图~\ref{fig:parallel1}
和图~\ref{fig:parallel2}。

\begin{figure}[!htbp]
  \centering
  \includegraphics{logo} \\
  此处可撰写对图的一些说明 \\
  支持多行
  \caption{此处是图的标题,也可以在此处撰写说明}\label{fig:single}
\end{figure}

\begin{figure}[!htbp]
  \centering
  \includegraphics[height=2cm]{logo}
  \hspace{1cm}
  \includegraphics[height=2cm]{logo}
  \caption{插入多个图形,共享同一个计数器}\label{fig:multi}
\end{figure}

\begin{figure}[!htbp]
  \begin{minipage}{0.45\textwidth}
    \centering
    \includegraphics[height=2cm]{logo}
    \caption{并排图一}\label{fig:parallel1}
  \end{minipage}
  \hfill
  \begin{minipage}{0.45\textwidth}
    \centering
    \includegraphics[height=2cm]{logo}
    \caption{并排图二}\label{fig:parallel2}
  \end{minipage}
\end{figure}

如果要为共用一个计数器的多个子图添加子标题,建议使用较新的 \pkg{subcaption} 宏包,
不建议使用 \pkg{subfigure} 或 \pkg{subfig} 等宏包。可以直接使用 \cs{subcaptionbox}
并排子图,或者使用 \cs{subcaption} 直接放在 minipage 中,用法同 \cs{caption}。
如图~\ref{fig:subcapbox},通过对 \cs{subcaptionbox} 和 \cs{includegraphics} 设置不同
的尺寸,即可实现两幅子图的任意的比例和尺寸。

\begin{figure}[!htbp]
  \centering
  \subcaptionbox{图一标题}[0.6\textwidth]{\includegraphics[height=2cm]{logo}}
  \subcaptionbox{图二标题}[0.3\textwidth]{\includegraphics[height=1cm]{logo}}
  \caption{同一个计数器包含多个子图}\label{fig:subcapbox}
\end{figure}

\subsection{表格}

根据印刷格式样式,表格中应使用小五号宋体,最好能够同时调整行间距和表格间距以获得较好的效果。
学术论文通常使用 \cs{toprule}、\cs{midrule}、\cs{bottomrule} 绘制三线表,效果如表~\ref{tab:sample} 所示。
此外,\pkg{threeparttable} 宏包提供了子环境可以支持在表格中添加脚注,\pkg{longtable} 宏包实现了可以书写跨页
的长表格。

\begin{table}[!htbp]
  \caption{这是一个标准的三线表}\label{tab:sample}
  \centering
  \footnotesize% fontsize,小五号字
  \setlength{\tabcolsep}{4pt}% column separation,表格间距
  \renewcommand{\arraystretch}{1.2}% row space ,行间距
  \begin{tabular}{lcccccccc}
      \toprule
      行号 & \multicolumn{8}{c}{跨多列的标题}\\
      \midrule
      Row 1 & $1$ & $2$ & $3$ & $4$ & $5$ & $6$ & $7$ & $8$\\
      Row 2 & $1$ & $2$ & $3$ & $4$ & $5$ & $6$ & $7$ & $8$\\
      Row 3 & $1$ & $2$ & $3$ & $4$ & $5$ & $6$ & $7$ & $8$\\
      \bottomrule
  \end{tabular}
\end{table}

\section{代码}

在论文中插入代码需要 \pkg{listings} 宏包,它支持代码的语法高亮等特性。config.tex 中配置了最简单的代码风格,
没有颜色,可以用于打印。

lstlisting 环境提供了可选参数,例如 captain 或者 title,前者可显示序号,而后者只显示标题而没有序号。
可选参数 label 提供了交叉引用的功能,例如代码~\ref{code}。

\begin{lstlisting}[language=C,caption=代码样例,label=code]
#include <stdio.h>
#include <unistd.h>
#include <sys/types.h>
#include <sys/wait.h>

int main() {
  pid_t pid;

  switch ((pid = fork())) {
  case -1:
    printf("fork failed\n");
    break;
  case 0:
    /* child calls exec */
    execl("/bin/ls", "ls", "-l", (char*)0);
    printf("execl failed\n");
    break;
  default:
    /* parent uses wait to suspend execution until child finishes */
    wait((int*)0);
    printf("is completed\n");
    break;
  }

  return 0;
}\end{lstlisting}

在 tex 文件中书写代码需要格外注意缩进,更加方便地是直接从代码源文件中导入代码片段,
即使用 \cs{lstinputlisting} 命令,它提供了参数 linerange 可以指定导入源文件的哪几行,
通过 \cs{lstname} 还可以直接获取文件名。

\lstinputlisting[linerange={10-17},language=C,caption=\lstname]{Source/code.cpp}

书写算法可以使用 \pkg{algorithm2e},如算法~\ref{algo},更多相关内容可查询相关文档。

\begin{algorithm}[!htbp]
    \SetAlgoLined\normalem\linespread{1.2}\small
    \caption{How to write algorithms}\label{algo}
    \KwData{this text}
    \KwResult{how to write algorithm with \LaTeX2e }
    initialization\;
    \While{not at end of this document}{
        read current\;
        \eIf{understand}{
            go to next section\;
            current section becomes this one\;
        }{
            go back to the beginning of current section\;
        }
    }
\end{algorithm}

\section{参考文献}

参考文献应符合国标 GB/T 7714-2015,该标准为替换印刷格式样式中规定的国标 GB/T 7714-2005 的最新版本,
使用 Bib\LaTeX{} 以及样式包 \pkg{biblatex-7714-2015} 可以轻松实现,后端使用 biber 引擎。

通常可以使用 \cs{cite} 直接引用,例如\cite{lamport1986document},参数可以添加多个,用逗号隔开,可以实现
同时引用多个参考文献\cite{chen2005zhulu,chu2004tushu,stamerjohanns2009mathml}。
使用 \cs{parencite} 可以获得非上标的以用\parencite{betts2005aging},同样可以同时引用
多个\parencite{bravo1990comparative,hls2012jinji,niu2013zonghe}。此外,还支持在脚注的引用,
这种方式的引用将与脚注共享编号,并且同时也会打印在最后的参考引用页中\footfullcite{chen1980zhongguo},打印在
参考引用页中的序号与脚注编号无关。

参考文献页\cite{yuan2012lana,yuan2012lanb,yuan2012lanc}将按照引用出现的顺序\footfullcite{walls2013drought}
生成序号\parencite{Bohan1928,Dubrovin1906}。

% \include{Tex/Chapter3}
%- 
\backmatter% 初始化其他部分环境,不建议注释
%-
\szubibliography% 导入参考文献
%-
\include{Tex/Appendix}% 导入附录
%-
%- 2020年新增,添加答辩记录,建议分成三个独立的PDF文件,
%- \szuaddpdf命令包含两个参数,[]中为可选参数,用于生成目录,{}中为PDF文件名,默认在Image下
% \szuaddpdf[指导教师对研究生学位论文的学术评语]{pingyu}
% \szuaddpdf[学位论文答辩委员会决议书]{dabian1}
%-\szuaddpdf{dabian2}% 前一页生成目录即可
%-
\chapter[致谢]{致\quad{}谢}

感谢国科大学位论文ucasthesis项目,以及上海交大学位论文SJTUThesis项目对开源的无私贡献。

感谢CTeX-kit项目对\LaTeX{}提供了中文支持。

本项目离不开\LaTeX{}开源社区的诸多宝贵资料,对\LaTeX{}开源社区各位贡献者表示感谢。

感谢参与测试和使用的各位同学。% 导入致谢
%-
\chapter{攻读硕士学位期间的研究成果}

%- 可以直接使用引用的格式

\begin{enumerate}[label = {[\arabic*]}]
    \item TANG T T. xxx xxxx xx xxx[M]. IEEE Transactions on xxxx. 2020-01-01.
    \item 唐同学. 一种基于某某的方法:中国,01100000.5[P]. 2020-01-01.
\end{enumerate}


%- 或者区分开

% \section*{论文}

% \begin{enumerate}[label = {[\arabic*]}]
%     \item TANG T T. xxx xxxx xx xxx. IEEE Transactions on xxxx. 2020.
% \end{enumerate}

% \section*{专利}

% \begin{enumerate}[label = {[\arabic*]}]
%     \item 唐同学. 一种基于某某的方法:中国,01100000.5. 2020-01-01.
% \end{enumerate}% 导入研究成果
%-
\end{document}
% %---------------------------------------------------------------------------%
