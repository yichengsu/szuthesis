\chapter{\LaTeX{} 入门}

\section{二级标题}

\subsection{三级标题}

\subsubsection{四级标题}

深圳大学1983年经国家教育部批准设立。中央、教育部和地方高度重视特区大学建设。北大援建中
文、外语类学科,清华援建电子、建筑类学科,人大援建经济、法律类学科,一大批知名学者云集
深圳大学。建校伊始,学校在高校管理体制上锐意改革,在奖学金、学分制、勤工俭学等方面进行
了积极探索,率先在国内实行毕业生不包分配和双向选择制度,推行教职员工全员聘任制度和后勤
部门社会化管理改革,在全国引起强烈反响\footnote{这是脚注}。

\section{字体}

字体部分主要分为正文字体和数学公式字体,这里主要展示正文字体,数学公式字体详见下节。

正文字体延续 ctex 关于字体的相关设置,需要注意的是,中文字体通常无粗体与斜体,例如
Windows 平台下,正文默认宋体,粗体则使用黑体代替,斜体则使用了楷体。MacOS 对多数
字体提供了完整的粗体表示,但是默认的斜体依然使用了楷体。通常论文的正文应简洁明了,
尽量避免使用多种字体。如需使用粗体或斜体,需重新定义默认字体族或者定义新的字体族。

由上段可知,字体的选择依赖于字库,而字库又依赖于系统设置,模板支持windows、adobe、mac、ubuntu
四种中文字库,通常情况下可根据系统类型自动选择。但是 Ubuntu 系统因缺少仿宋字体,而部分章节的标题
会使用到仿宋,模板使用宋体进行了代替,依然推荐 Ubuntu 用户安装完整 adobe 字体并且指定字库。

关于字库的显示效果可参考表~\ref{tab:fontset},该表展示了 Windows 系统下主要的字体以及对应的粗体与楷体的
兼容情况。可以看到默认的粗体使用了黑体代替,而斜体则使用了楷书。中文字体的设置均不影响英文字体的设置,英文
字体除摘要页标题使用了 Lucida Console 字体外,全局使用开源 Times New Roman 字体 XITS。可以看到 \cs{textit}
和 \cs{kaishu} 都可以生成中文的楷书,但是对于英文字体却产生了不同的影响,书写时需注意。

\def\SZUFONTEG{深圳大学fgFG456}
\begin{table}
  \centering
  \caption{字体选择}\label{tab:fontset}
  \begin{tabular}{lcccc}
    \toprule
    字体 & 命令          & 常规                  & 粗体                         & 斜体                         \\
    \midrule
    宋体 & -             & {\SZUFONTEG}          & \textbf{\SZUFONTEG}          & \textit{\SZUFONTEG}          \\
         & \cs{textrm}   & \textrm{\SZUFONTEG}   & \textbf{\textrm{\SZUFONTEG}} & \textit{\textrm{\SZUFONTEG}} \\
         & \cs{songti}   & {\songti\SZUFONTEG}   & -                            & -                            \\
    黑体 & \cs{textsf}   & \textsf{\SZUFONTEG}   & \textbf{\textsf{\SZUFONTEG}} & -                            \\
         & \cs{heiti}    & {\heiti\SZUFONTEG}    & -                            & -                            \\
    仿宋 & \cs{texttt}   & \texttt{\SZUFONTEG}   & -                            & -                            \\
         & \cs{fangsong} & {\fangsong\SZUFONTEG} & -                            & -                            \\
    楷书 & \cs{textit}   & \textit{\SZUFONTEG}   & -                            & -                            \\
         & \cs{kaishu}   & {\kaishu\SZUFONTEG}   & -                            & -                            \\
    \bottomrule
  \end{tabular}
\end{table}

此外,windows 和 mac 对应的中文字库还提供了微软雅黑 \cs{yahei}、隶书 \cs{lishu}、幼圆 \cs{youyuan} 三种字体,
\cs{yahei} 默认提供了粗体。更多关于默认字体与字体库的选择相关问题可查阅 ctex 文档 4.3 节中文字库。

除了中文字库外,由 XITS 字库可支持俄语 Привет ,默认的中文字体宋体通常支持日语 こんにちは 。
如需要韩语,则需要导入新的字库,如 Windows 通常带有 ARIALUNI.TTF 韩语字库,可以使用
\lstinline[language=Tex]!\newCJKfontfamily{\hanyu}{ARIALUNI.TTF}! 导入新的字库,就可以使用 \cs{hanyu} 来
书写韩语了。其他语言同理,可以使用这个命令导入对应的字库。

\section{数学}

\subsection{数学符号与公式}

数学公式均统一使用了开源 Times New Roman 字体 XITS MATH,这套字体除了基本的数字与字母外,还额外设计了许多
数学符号,包括全局粗体。使用这套字体需导入 \pkg{unicode-math},所以在启用数学相关字体时需要注意,使用 \cs{sym..} 系列
命令代替 \cs{math..} 系列命令,以获得统一的字体风格,\cs{math..} 系列命令将会使用普通字母的字体 XITS,而不是
对应的 XITS MATH。例如加粗 \cs{mathbf} 将使用 \cs{symbf} 代替。
\[
  \begin{split}
    \symcal{L} \{f\}(s) &= \int _{0^{-}}^{\infty} f(t) e^{-st} \, \symrm{d}t, \
    \symscr{L} \{f\}(s) = \int _{0^{-}}^{\infty} f(t) e^{-st} \, \symrm{d}t\\
    \symcal{F} \bigl( f(x+x_{0}) \bigr) &= \symcal{F} \bigl( f(x) \bigr) e^{2\pi i\xi x_{0}}, \
    \symscr{F} \bigl( f(x+x_{0}) \bigr) = \symscr{F} \bigl( f(x) \bigr) e^{2\pi i\xi x_{0}}
  \end{split}
\]
\begin{equation}\label{equ:1}
  \begin{cases}
    \frac{\partial \rho}{\partial t} + \nabla\cdot(\rho\Vector{V}) = 0                                                \\
    \frac{\partial (\rho\Vector{V})}{\partial t} + \nabla\cdot(\rho\Vector{V}\Vector{V}) = \nabla\cdot\Tensor{\sigma} \\
    \frac{\partial (\rho E)}{\partial t} + \nabla\cdot(\rho E\Vector{V}) = \nabla\cdot(k\nabla T) + \nabla\cdot(\Tensor{\sigma}\cdot\Vector{V})
  \end{cases}
\end{equation}
\begin{equation}\label{equ:2}
  \frac{\partial }{\partial t}\int\limits_{\Omega} u \, \symrm{d}\Omega + \int\limits_{S} \Vector{n}\cdot(u\Vector{V}) \, \symrm{d}S = \dot{\phi}
\end{equation}

表~\ref{tab:mathfontset} 中简要示范了几种常用的数学字体,可以看到指令并不是对所有的数字和希腊字母都有效。更多关于
数学字体相关的知识推荐查阅《在 \LaTeX{} 中使用 OpenType 字体》\footnote{\url{https://stone-zeng.github.io/2020-05-02-use-opentype-fonts-iii/}\label{stonezeng}}
一文,该文中``字符支持''一节详细介绍了各个命令的使用场景。

\def\SZUMATHFONTEG{fgFG456\delta\Delta}
\begin{table}[!htbp]
  \centering
  \caption{数学公式字体选择}\label{tab:mathfontset}
  \begin{tabular}{cccccc}
    \toprule
    命令        & 样例                        & 命令         & 样例                         & 命令         & 样例                         \\
    \midrule
    -           & \({\SZUMATHFONTEG}\)        & \cs{symbf}   & \(\symbf{\SZUMATHFONTEG}\)   & \cs{symrm}   & \(\symrm{\SZUMATHFONTEG}\)   \\
    \cs{symup}  & \(\symup{\SZUMATHFONTEG}\)  & \cs{symbfup} & \(\symbfup{\SZUMATHFONTEG}\) & \cs{symit}   & \(\symit{\SZUMATHFONTEG}\)   \\
    \cs{symbb}  & \(\symbb{\SZUMATHFONTEG}\)  & \cs{symsfup} & \(\symsfup{\SZUMATHFONTEG}\) & \cs{symtt}   & \(\symtt{\SZUMATHFONTEG}\)   \\
    \cs{symcal} & \(\symcal{\SZUMATHFONTEG}\) & \cs{symscr}  & \(\symscr{\SZUMATHFONTEG}\)  & \cs{symfrak} & \(\symfrak{\SZUMATHFONTEG}\) \\
    \bottomrule
  \end{tabular}
\end{table}

在使用数学字体过程中,应遵循相应的标准,根据 GB 3102.11《物理科学和技术中使用的数学
符号》\footnote{\url{https://zh.wikisource.org/wiki/GB_3102.11}}中规定:

\begin{itemize}
  \small
  \item 变量 (例如 \(x\), \(y\) 等)、变动附标 (例如 \(\sum_i x_i\) 中的 \(i\)) 及函数 (例如 \(f\), \(g\) 等)
        用斜体字母表示。点 \(A\)、线段 \(AB\) 及弧 \(CD\) 用斜体字母表示。
        在特定场合中视为常数的参数 (例如 \(a\), \(b\) 等) 也用斜体字母表示。
  \item 有定义的已知函数 (例如 \(\sin \),\(\exp \),\(\ln \),\(\Gamma \) 等) 用正体字母表示。
  其值不变的数学常数 (例如 \(\exp = 2.718 281 8\ldots \),\(\symrm{\pi} = 3.141 592 6\ldots \),\(\symrm{i}^2 = −1\) 等) 用正体
  字母表示。已定义的算子 (例如 \(\Div \),\(\deltarm x\) 中的 \(\deltarm \) 及 \(\Dif 𝑓/\Dif 𝑥\) 中的 \(\Dif \)) 也用正体字母表示。
  \item 数字表中数 (例如 351 204,1.32,7/8) 的表示用正体。
\end{itemize}

根据这一规定,在 config.tex 中额外定义了一些通用命令,如表~\ref{tab:mathcmd} 所示。使用这些命令,
我们可以方便的定义公式,如公式\ref{equ:1}、\ref{equ:2}和公式~\ref{equ:3}\textsuperscript{\ref{stonezeng}}。
\begin{equation}\label{equ:3}
  \begin{aligned}
    \oiint_{\partial\Omega} \Vector{}{E} \cdot \Dif \Vector{S} & = \frac{1}{\epsilon_0} \iiint_\Omega \rho \, \Dif V & \quad &
    \oint_{\partial\Sigma} \Vector{E} \cdot \Dif \Vector{l} = -\frac{\Dif}{\Dif t} \iint_\Sigma \Vector{B} \cdot \Dif \Vector{S}         \\
    \oiint_{\partial\Omega} \Vector{B} \cdot \Dif \Vector{S} & = 0                                                & \quad &
    \oint_{\partial\Sigma} \Vector{B} \cdot \Dif \Vector{l} = \mu_0 \iint_\Sigma \Vector{J} \cdot \Vector{S}
    + \mu_0 \epsilon_0 \frac{\Dif}{\Dif t} \iint_\Sigma \Vector{E} \cdot \Dif \Vector{S} \\
  \end{aligned}
\end{equation}

\begin{table}[!htbp]
  \centering
  \caption{自定义指令}\label{tab:mathcmd}
  \begin{tabular}{cccccccc}
    \toprule
    用法                   & 样例            & 用法                   & 样例           & 用法                   & 样例           & 用法       & 样例        \\
    \midrule
    \verb!\Vector{A}! & \(\Vector{A}\)  & \verb!\Matrix{A}! & \(\Matrix{A}\) & \verb!\Tensor{A}! & \(\Tensor{A}\) & \cs{Dif x} & \(\Dif x\)  \\
    \verb!\Const{\pi}! & \(\Const{\pi}\) & \cs{deltarm}           & \(\deltarm \)  & \cs{Div}               & \(\Div \)      & \cs{Trace} & \(\Trace \) \\
    \bottomrule
  \end{tabular}
\end{table}

\subsection{标题中带有公式 \texorpdfstring{\(\Lambda,\lambda,\theta,\bar{\Lambda},\sqrt{S_{NN}}\)}{}}

\LaTeX{} 支持在标题中带有公式,以及生成的目录中带有公式。如本节标题所示,在目录中均可显示完整的公式。
由于 \pkg{hyperref} 在生成书签时不支持公式,所以当使用该包时,可以使用 \cs{texorpdfstring} 命令。
这个命令的第一个参数可以书写公式,第二个参数则用于生成书签,可以使用 \cs{textLambda} 等书签兼容的公式,
不过最简单的方法就是空出第二个参数,因为书签在打印时不显示,所以无需额外关注。

\subsection{数学环境}
以下环境共享一个计数器,如果取消这一设定,在 config.tex 中删除 \cs{newtheorem} 的第二个参数[theorem]即可。

\begin{theorem}      这是一个定理。 \end{theorem}
\begin{axiom}        这是一个公理。 \end{axiom}
\begin{lemma}        这是一个引理。 \end{lemma}
\begin{corollary}    这是一个推论。 \end{corollary}
\begin{assertion}    这是一个断言。 \end{assertion}
\begin{proposition}  这是一个命题。 \end{proposition}
\begin{conjecture}   这是一个猜想。 \end{conjecture}
\begin{assumption}   这是一个假设。 \end{assumption}

以下环境均为独立的计数器。

\begin{definition}   这是一个定义。 \end{definition}
\begin{example}      这是一个例子。 \end{example}
\begin{problem}      这是一个问题。 \end{problem}
\begin{exercise}     这是一个练习。 \end{exercise}
\begin{remark}       这是一个注。   \end{remark}
\begin{solution}     这是一个解。   \end{solution}
\begin{proof}        这是一个证明。 \end{proof}

\section{浮动体}

\subsection{图形}

插入单个图形时效果如图~\ref{fig:single}。同样的方法可以插入多个图形,通过可选参数可以指定图形的高度
或者宽度,效果如图~\ref{fig:multi}。如果各个子图相互独立,可以使用 \pkg{minipage},如图~\ref{fig:parallel1}
和图~\ref{fig:parallel2}。

\begin{figure}[!htbp]
  \centering
  \includegraphics{logo} \\
  此处可撰写对图的一些说明 \\
  支持多行
  \caption{此处是图的标题,也可以在此处撰写说明}\label{fig:single}
\end{figure}

\begin{figure}[!htbp]
  \centering
  \includegraphics[height=2cm]{logo}
  \hspace{1cm}
  \includegraphics[height=2cm]{logo}
  \caption{插入多个图形,共享同一个计数器}\label{fig:multi}
\end{figure}

\begin{figure}[!htbp]
  \begin{minipage}{0.45\textwidth}
    \centering
    \includegraphics[height=2cm]{logo}
    \caption{并排图一}\label{fig:parallel1}
  \end{minipage}
  \hfill
  \begin{minipage}{0.45\textwidth}
    \centering
    \includegraphics[height=2cm]{logo}
    \caption{并排图二}\label{fig:parallel2}
  \end{minipage}
\end{figure}

如果要为共用一个计数器的多个子图添加子标题,建议使用较新的 \pkg{subcaption} 宏包,
不建议使用 \pkg{subfigure} 或 \pkg{subfig} 等宏包。可以直接使用 \cs{subcaptionbox}
并排子图,或者使用 \cs{subcaption} 直接放在 minipage 中,用法同 \cs{caption}。
如图~\ref{fig:subcapbox},通过对 \cs{subcaptionbox} 和 \cs{includegraphics} 设置不同
的尺寸,即可实现两幅子图的任意的比例和尺寸。

\begin{figure}[!htbp]
  \centering
  \subcaptionbox{图一标题}[0.6\textwidth]{\includegraphics[height=2cm]{logo}}
  \subcaptionbox{图二标题}[0.3\textwidth]{\includegraphics[height=1cm]{logo}}
  \caption{同一个计数器包含多个子图}\label{fig:subcapbox}
\end{figure}

\subsection{表格}

根据印刷格式样式,表格中应使用小五号宋体,最好能够同时调整行间距和表格间距以获得较好的效果。
学术论文通常使用 \cs{toprule}、\cs{midrule}、\cs{bottomrule} 绘制三线表,效果如表~\ref{tab:sample} 所示。
此外,\pkg{threeparttable} 宏包提供了子环境可以支持在表格中添加脚注,\pkg{longtable} 宏包实现了可以书写跨页
的长表格。

\begin{table}[!htbp]
  \caption{这是一个标准的三线表}\label{tab:sample}
  \centering
  \footnotesize% fontsize,小五号字
  \setlength{\tabcolsep}{4pt}% column separation,表格间距
  \renewcommand{\arraystretch}{1.2}% row space ,行间距
  \begin{tabular}{lcccccccc}
      \toprule
      行号 & \multicolumn{8}{c}{跨多列的标题}\\
      \midrule
      Row 1 & $1$ & $2$ & $3$ & $4$ & $5$ & $6$ & $7$ & $8$\\
      Row 2 & $1$ & $2$ & $3$ & $4$ & $5$ & $6$ & $7$ & $8$\\
      Row 3 & $1$ & $2$ & $3$ & $4$ & $5$ & $6$ & $7$ & $8$\\
      \bottomrule
  \end{tabular}
\end{table}

\section{代码}

在论文中插入代码需要 \pkg{listings} 宏包,它支持代码的语法高亮等特性。config.tex 中配置了最简单的代码风格,
没有颜色,可以用于打印。

lstlisting 环境提供了可选参数,例如 captain 或者 title,前者可显示序号,而后者只显示标题而没有序号。
可选参数 label 提供了交叉引用的功能,例如代码~\ref{code}。

\begin{lstlisting}[language=C,caption=代码样例,label=code]
#include <stdio.h>
#include <unistd.h>
#include <sys/types.h>
#include <sys/wait.h>

int main() {
  pid_t pid;

  switch ((pid = fork())) {
  case -1:
    printf("fork failed\n");
    break;
  case 0:
    /* child calls exec */
    execl("/bin/ls", "ls", "-l", (char*)0);
    printf("execl failed\n");
    break;
  default:
    /* parent uses wait to suspend execution until child finishes */
    wait((int*)0);
    printf("is completed\n");
    break;
  }

  return 0;
}\end{lstlisting}

在 tex 文件中书写代码需要格外注意缩进,更加方便地是直接从代码源文件中导入代码片段,
即使用 \cs{lstinputlisting} 命令,它提供了参数 linerange 可以指定导入源文件的哪几行,
通过 \cs{lstname} 还可以直接获取文件名。

\lstinputlisting[linerange={10-17},language=C,caption=\lstname]{Source/code.cpp}

书写算法可以使用 \pkg{algorithm2e},如算法~\ref{algo},更多相关内容可查询相关文档。

\begin{algorithm}[!htbp]
    \SetAlgoLined\normalem\linespread{1.2}\small
    \caption{How to write algorithms}\label{algo}
    \KwData{this text}
    \KwResult{how to write algorithm with \LaTeX2e }
    initialization\;
    \While{not at end of this document}{
        read current\;
        \eIf{understand}{
            go to next section\;
            current section becomes this one\;
        }{
            go back to the beginning of current section\;
        }
    }
\end{algorithm}

\section{参考文献}

参考文献应符合国标 GB/T 7714-2015,该标准为替换印刷格式样式中规定的国标 GB/T 7714-2005 的最新版本,
使用 Bib\LaTeX{} 以及样式包 \pkg{biblatex-7714-2015} 可以轻松实现,后端使用 biber 引擎。

通常可以使用 \cs{cite} 直接引用,例如\cite{lamport1986document},参数可以添加多个,用逗号隔开,可以实现
同时引用多个参考文献\cite{chen2005zhulu,chu2004tushu,stamerjohanns2009mathml}。
使用 \cs{parencite} 可以获得非上标的以用\parencite{betts2005aging},同样可以同时引用
多个\parencite{bravo1990comparative,hls2012jinji,niu2013zonghe}。此外,还支持在脚注的引用,
这种方式的引用将与脚注共享编号,并且同时也会打印在最后的参考引用页中\footfullcite{chen1980zhongguo},打印在
参考引用页中的序号与脚注编号无关。

参考文献页\cite{yuan2012lana,yuan2012lanb,yuan2012lanc}将按照引用出现的顺序\footfullcite{walls2013drought}
生成序号\parencite{Bohan1928,Dubrovin1906}。
